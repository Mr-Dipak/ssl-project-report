% Appendix C

\chapter{Additional Documentation} % Main appendix title

\label{AppendixC} % For referencing this appendix elsewhere, use \ref{AppendixC}

% Header formatting is now handled in main.tex

This appendix contains additional technical documentation, API references, and deployment guides that support the SSL Automation Tool implementation.

\section{API Documentation}

\subsection{REST API Endpoints}

The SSL Automation Tool provides a comprehensive REST API for certificate management operations.

\subsubsection{Certificate Generation}

\textbf{Endpoint}: \texttt{POST /api/v1/generate-cert}

\textbf{Description}: Initiates SSL certificate generation using Let's Encrypt with DNS validation.

\textbf{Request Body}:
\begin{verbatim}
{
  "domain": "example.com",
  "email": "admin@example.com",
  "dns_provider": "godaddy",
  "api_key": "your_godaddy_api_key",
  "api_secret": "your_godaddy_api_secret",
  "base_domain": "example.com",
  "ttl": 600,
  "propagation_delay": 60
}
\end{verbatim}

\textbf{Response}:
\begin{verbatim}
{
  "success": true,
  "message": "Certificate generated successfully",
  "domain": "example.com",
  "output": "Certbot execution output..."
}
\end{verbatim}

\section{Environment Configuration}

\subsection{Required Environment Variables}

The following environment variables must be configured for proper system operation:

\subsubsection{Frontend Configuration}

\begin{verbatim}
# .env.local - Frontend environment variables
NEXT_PUBLIC_API_URL=http://localhost:80
NEXT_PUBLIC_ENV=development
\end{verbatim}

\subsubsection{Backend Configuration}

Environment variables are passed at runtime through Docker or system configuration:

\begin{verbatim}
# Runtime environment variables
GODADDY_API_KEY=your_api_key_here
GODADDY_API_SECRET=your_api_secret_here
BASE_DOMAIN=yourdomain.com
DNS_TTL=600
PROPAGATION_DELAY=60
PYTHONPATH=/app
\end{verbatim}

\section{Security Considerations}

\subsection{Security Best Practices}

\subsubsection{Credential Management}

\begin{itemize}
    \item Never store DNS provider credentials in configuration files
    \item Use environment variables or secrets management systems
    \item Rotate API keys regularly (quarterly recommended)
    \item Monitor API key usage for unauthorized access
    \item Implement IP whitelisting where possible
\end{itemize}

\subsubsection{Certificate Security}

\begin{itemize}
    \item Set proper file permissions (600) for private keys
    \item Use encrypted storage for certificate backups
    \item Implement certificate pinning for critical applications
    \item Monitor certificate expiration proactively
    \item Have emergency certificate renewal procedures
\end{itemize}

