% Appendix B

\chapter{Sample Program Code} % Main appendix title

\label{AppendixB} % For referencing this appendix elsewhere, use \ref{AppendixB}

% Header formatting is now handled in main.tex

This appendix contains key program code samples that demonstrate the core functionality and implementation details of the SSL Automation Tool.

\section{Frontend Code Samples}

\subsection{SSL Dashboard Component}

The main dashboard component that manages certificate display and operations:

\begin{verbatim}
// SSLDashboard.tsx - Main certificate management interface
import React, { useState, useEffect } from 'react';
import { Card, CardContent, CardHeader, CardTitle } from '@/components/ui/card';
import { Button } from '@/components/ui/button';
import { SSLApiService } from '@/lib/api';
import { Certificate } from '@/types/ssl';

export function SSLDashboard() {
  const [certificates, setCertificates] = useState<Certificate[]>([]);
  const [loading, setLoading] = useState(true);

  useEffect(() => {
    loadCertificates();
  }, []);

  const loadCertificates = async () => {
    try {
      setLoading(true);
      const response = await SSLApiService.listCertificates();
      setCertificates(response.certificates);
    } catch (error) {
      console.error('Failed to load certificates:', error);
    } finally {
      setLoading(false);
    }
  };

  const downloadCertificate = async (domain: string, type: string) => {
    try {
      const blob = await SSLApiService.downloadCertificateFile(domain, type);
      SSLApiService.triggerDownload(blob, `${domain}-${type}.pem`);
    } catch (error) {
      console.error('Download failed:', error);
    }
  };

  const getStatusColor = (cert: Certificate) => {
    if (!cert.valid) return 'red';
    const daysUntilExpiry = Math.ceil(
      (new Date(cert.expires_on).getTime() - Date.now()) / (1000 * 60 * 60 * 24)
    );
    if (daysUntilExpiry < 7) return 'red';
    if (daysUntilExpiry < 30) return 'yellow';
    return 'green';
  };

  return (
    <div className="space-y-6">
      <div className="grid grid-cols-1 md:grid-cols-4 gap-4">
        {/* Statistics Cards */}
        <Card>
          <CardHeader>
            <CardTitle>Total Certificates</CardTitle>
          </CardHeader>
          <CardContent>
            <div className="text-2xl font-bold">{certificates.length}</div>
          </CardContent>
        </Card>
        {/* Additional statistics cards... */}
      </div>
      
      {/* Certificate List */}
      <Card>
        <CardHeader>
          <CardTitle>Certificate Management</CardTitle>
        </CardHeader>
        <CardContent>
          {certificates.map((cert) => (
            <div key={cert.domain} className="border-b py-4">
              <div className="flex justify-between items-center">
                <div>
                  <h3 className="font-medium">{cert.domain}</h3>
                  <p className="text-sm text-gray-500">
                    Expires: {new Date(cert.expires_on).toLocaleDateString()}
                  </p>
                </div>
                <div className="flex space-x-2">
                  <Button
                    onClick={() => downloadCertificate(cert.domain, 'fullchain')}
                    size="sm"
                  >
                    Download Fullchain
                  </Button>
                  <Button
                    onClick={() => downloadCertificate(cert.domain, 'privkey')}
                    size="sm"
                  >
                    Download Private Key
                  </Button>
                </div>
              </div>
            </div>
          ))}
        </CardContent>
      </Card>
    </div>
  );
}
\end{verbatim}

\subsection{Certificate Generation Form}

Form component for creating new SSL certificates:

\begin{verbatim}
// CertificateGenerationDialog.tsx - Certificate creation interface
import React from 'react';
import { useForm } from 'react-hook-form';
import { zodResolver } from '@hookform/resolvers/zod';
import * as z from 'zod';
import { Button } from '@/components/ui/button';
import { Input } from '@/components/ui/input';
import { Label } from '@/components/ui/label';
import { SSLApiService } from '@/lib/api';

const formSchema = z.object({
  domain: z.string().min(1, 'Domain is required'),
  email: z.string().email('Valid email is required'),
  dns_provider: z.literal('godaddy'),
  api_key: z.string().min(1, 'API key is required'),
  api_secret: z.string().min(1, 'API secret is required'),
  base_domain: z.string().min(1, 'Base domain is required'),
  ttl: z.number().min(300).max(86400).optional(),
  propagation_delay: z.number().min(30).max(600).optional(),
});

type FormData = z.infer<typeof formSchema>;

export function CertificateGenerationDialog({ onSuccess, onClose }) {
  const [isGenerating, setIsGenerating] = useState(false);
  
  const form = useForm<FormData>({
    resolver: zodResolver(formSchema),
    defaultValues: {
      dns_provider: 'godaddy',
      ttl: 600,
      propagation_delay: 60,
    },
  });

  const onSubmit = async (data: FormData) => {
    try {
      setIsGenerating(true);
      const response = await SSLApiService.generateCertificate(data);
      
      if (response.success) {
        toast.success('Certificate generated successfully!');
        onSuccess();
        onClose();
      } else {
        toast.error(`Generation failed: ${response.error}`);
      }
    } catch (error) {
      toast.error('Failed to generate certificate');
      console.error(error);
    } finally {
      setIsGenerating(false);
    }
  };

  return (
    <form onSubmit={form.handleSubmit(onSubmit)} className="space-y-4">
      <div>
        <Label htmlFor="domain">Domain</Label>
        <Input
          id="domain"
          placeholder="example.com or *.example.com"
          {...form.register('domain')}
        />
        {form.formState.errors.domain && (
          <p className="text-red-500 text-sm">
            {form.formState.errors.domain.message}
          </p>
        )}
      </div>

      <div>
        <Label htmlFor="email">Email</Label>
        <Input
          id="email"
          type="email"
          placeholder="admin@example.com"
          {...form.register('email')}
        />
      </div>

      <div>
        <Label htmlFor="api_key">GoDaddy API Key</Label>
        <Input
          id="api_key"
          type="password"
          {...form.register('api_key')}
        />
      </div>

      <div>
        <Label htmlFor="api_secret">GoDaddy API Secret</Label>
        <Input
          id="api_secret"
          type="password"
          {...form.register('api_secret')}
        />
      </div>

      <Button 
        type="submit" 
        disabled={isGenerating}
        className="w-full"
      >
        {isGenerating ? 'Generating...' : 'Generate Certificate'}
      </Button>
    </form>
  );
}
\end{verbatim}

\section{Backend Code Samples}

\subsection{FastAPI Main Application}

The core FastAPI application setup with CORS and routing:

\begin{verbatim}
# app/main.py - FastAPI application entry point
from fastapi import FastAPI
from fastapi.middleware.cors import CORSMiddleware
from app.routes import certs

app = FastAPI(
    title="SSL Automation API",
    description="Automated SSL certificate generation using Let's Encrypt",
    version="1.0.0"
)

# CORS configuration for frontend integration
app.add_middleware(
    CORSMiddleware,
    allow_origins=[
        "http://localhost:3000",
        "http://localhost:3001",
        "http://127.0.0.1:3000",
        "http://127.0.0.1:3001"
    ],
    allow_credentials=True,
    allow_methods=["*"],
    allow_headers=["*"],
)

# Include certificate management routes
app.include_router(certs.router, prefix="/api/v1")

@app.get("/")
async def root():
    return {"message": "SSL Automation API is running"}

@app.get("/health")
async def health_check():
    return {"status": "healthy", "service": "ssl-automation-api"}
\end{verbatim}

\subsection{Certificate Service Implementation}

Core business logic for certificate operations:

\begin{verbatim}
# app/services/cert_service.py - Certificate file operations
import os
import zipfile
from pathlib import Path
from typing import Optional, Dict, List
from app.utils.cert_info import get_cert_info

class CertService:
    @staticmethod
    def get_cert_paths(domain: str) -> Optional[Dict[str, str]]:
        """Get paths to certificate files for a domain."""
        cert_dir = Path(f"/etc/letsencrypt/live/{domain}")
        
        if not cert_dir.exists():
            return None
            
        fullchain_path = cert_dir / "fullchain.pem"
        privkey_path = cert_dir / "privkey.pem"
        
        if fullchain_path.exists() and privkey_path.exists():
            return {
                "fullchain": str(fullchain_path),
                "privkey": str(privkey_path)
            }
        
        return None
    
    @staticmethod
    def create_cert_zip(domain: str) -> Optional[str]:
        """Create a ZIP archive containing all certificate files."""
        cert_paths = CertService.get_cert_paths(domain)
        if not cert_paths:
            return None
            
        zip_path = f"/tmp/{domain}-certificates.zip"
        
        with zipfile.ZipFile(zip_path, 'w') as zipf:
            for file_type, file_path in cert_paths.items():
                if os.path.exists(file_path):
                    zipf.write(file_path, f"{file_type}.pem")
        
        return zip_path if os.path.exists(zip_path) else None
    
    @staticmethod
    def list_all_certificates() -> List[Dict]:
        """List all managed certificates with their information."""
        certificates = []
        le_dir = Path("/etc/letsencrypt/live")
        
        if not le_dir.exists():
            return certificates
            
        for domain_dir in le_dir.iterdir():
            if domain_dir.is_dir():
                cert_info = get_cert_info(domain_dir.name)
                if cert_info:
                    certificates.append(cert_info)
                    
        return certificates
\end{verbatim}

\subsection{Certbot Runner Service}

Service for executing Certbot commands with DNS validation:

\begin{verbatim}
# app/services/certbot_runner.py - Certbot execution wrapper
import subprocess
import os
import logging
from typing import Dict, Tuple

logger = logging.getLogger(__name__)

class CertbotRunner:
    @staticmethod
    def generate_certificate(
        domain: str,
        email: str,
        dns_provider: str,
        api_key: str,
        api_secret: str,
        base_domain: str,
        ttl: int = 600,
        propagation_delay: int = 60
    ) -> Tuple[bool, str]:
        """Execute Certbot to generate SSL certificate."""
        
        # Set environment variables for DNS hook
        env = os.environ.copy()
        env.update({
            'GODADDY_API_KEY': api_key,
            'GODADDY_API_SECRET': api_secret,
            'BASE_DOMAIN': base_domain,
            'DNS_TTL': str(ttl),
            'PROPAGATION_DELAY': str(propagation_delay)
        })
        
        # Construct Certbot command
        cmd = [
            'certbot', 'certonly',
            '--manual',
            '--preferred-challenges=dns',
            '--manual-auth-hook', '/app/scripts/godaddy_auth_hook.py',
            '--manual-cleanup-hook', '/app/scripts/godaddy_cleanup_hook.py',
            '--domain', domain,
            '--email', email,
            '--agree-tos',
            '--non-interactive',
            '--manual-public-ip-logging-ok'
        ]
        
        try:
            logger.info(f"Starting certificate generation for {domain}")
            
            result = subprocess.run(
                cmd,
                env=env,
                capture_output=True,
                text=True,
                timeout=300  # 5 minute timeout
            )
            
            if result.returncode == 0:
                logger.info(f"Certificate generated successfully for {domain}")
                return True, result.stdout
            else:
                logger.error(f"Certbot failed for {domain}: {result.stderr}")
                return False, result.stderr
                
        except subprocess.TimeoutExpired:
            logger.error(f"Certbot timeout for {domain}")
            return False, "Certificate generation timed out"
        except Exception as e:
            logger.error(f"Certbot execution failed for {domain}: {str(e)}")
            return False, f"Execution failed: {str(e)}"
\end{verbatim}

\section{DNS Integration Code}

\subsection{GoDaddy DNS Hook Script}

Python script for automated DNS challenge handling:

\begin{verbatim}
#!/usr/bin/env python3
# app/scripts/godaddy_auth_hook.py - GoDaddy DNS authentication hook

import os
import sys
import requests
import time
import logging

logging.basicConfig(level=logging.INFO)
logger = logging.getLogger(__name__)

def create_txt_record():
    """Create TXT record for DNS challenge validation."""
    
    # Get environment variables set by Certbot
    domain = os.environ.get('CERTBOT_DOMAIN')
    validation = os.environ.get('CERTBOT_VALIDATION')
    
    # Get GoDaddy API credentials
    api_key = os.environ.get('GODADDY_API_KEY')
    api_secret = os.environ.get('GODADDY_API_SECRET')
    base_domain = os.environ.get('BASE_DOMAIN')
    ttl = int(os.environ.get('DNS_TTL', 600))
    delay = int(os.environ.get('PROPAGATION_DELAY', 60))
    
    if not all([domain, validation, api_key, api_secret, base_domain]):
        logger.error("Missing required environment variables")
        sys.exit(1)
    
    # Construct record name
    record_name = f"_acme-challenge.{domain}"
    if domain.startswith('*.'):
        record_name = f"_acme-challenge.{domain[2:]}"
    
    # Remove base domain from record name for GoDaddy API
    if record_name.endswith(f".{base_domain}"):
        record_name = record_name[:-len(f".{base_domain}")]
    
    # GoDaddy API headers
    headers = {
        'Authorization': f'sso-key {api_key}:{api_secret}',
        'Content-Type': 'application/json'
    }
    
    # Record data
    record_data = [{
        'data': validation,
        'ttl': ttl
    }]
    
    # API endpoint
    url = f"https://api.godaddy.com/v1/domains/{base_domain}/records/TXT/{record_name}"
    
    try:
        logger.info(f"Creating TXT record: {record_name} = {validation}")
        
        response = requests.put(url, json=record_data, headers=headers)
        response.raise_for_status()
        
        logger.info(f"TXT record created successfully")
        logger.info(f"Waiting {delay} seconds for DNS propagation...")
        
        time.sleep(delay)
        
        logger.info("DNS propagation wait complete")
        
    except requests.exceptions.RequestException as e:
        logger.error(f"Failed to create TXT record: {e}")
        sys.exit(1)

if __name__ == "__main__":
    create_txt_record()
\end{verbatim}

\section{API Models and Types}

\subsection{Pydantic Models}

Data validation models for API requests and responses:

\begin{verbatim}
# app/models/models.py - Pydantic data models
from pydantic import BaseModel, EmailStr
from typing import Optional, List
from datetime import datetime

class CertRequest(BaseModel):
    """Certificate generation request model."""
    domain: str
    email: EmailStr
    dns_provider: str = "godaddy"
    api_key: str
    api_secret: str
    base_domain: str
    ttl: Optional[int] = 600
    propagation_delay: Optional[int] = 60

class CertFetchRequest(BaseModel):
    """Certificate retrieval request model."""
    domain: str

class Certificate(BaseModel):
    """Certificate information model."""
    domain: str
    issued_on: str
    expires_on: str
    valid: bool
    fullchain_path: Optional[str] = None
    privkey_path: Optional[str] = None

class CertificateListResponse(BaseModel):
    """Certificate list response model."""
    certificates: List[Certificate]
    total: int

class CertificateGenerationResponse(BaseModel):
    """Certificate generation response model."""
    success: bool
    message: str
    domain: Optional[str] = None
    error: Optional[str] = None
    output: Optional[str] = None
\end{verbatim}

\section{Configuration Files}

\subsection{Docker Configuration}

Docker Compose setup for containerized deployment:

\begin{verbatim}
# docker-compose.yml - Container orchestration
version: '3.8'

services:
  ssl-automation-backend:
    build: .
    ports:
      - "80:8000"
    volumes:
      - .:/app
      - cert-data:/etc/letsencrypt
    environment:
      - PYTHONPATH=/app
    working_dir: /app
    command: uv run uvicorn app.main:app --host 0.0.0.0 --port 8000

volumes:
  cert-data:
    driver: local
\end{verbatim}

\subsection{Python Project Configuration}

Modern Python project setup with UV package manager:

\begin{verbatim}
# pyproject.toml - Python project configuration
[project]
name = "ssl-automation-backend"
version = "0.1.0"
description = "Automated SSL certificate generation using Let's Encrypt"
readme = "README.md"
requires-python = ">=3.13"

dependencies = [
    "fastapi>=0.115.12",
    "uvicorn>=0.34.3",
    "pydantic>=2.5.0",
    "pydantic[email]>=2.5.0",
    "certbot>=4.1.1",
    "cryptography>=45.0.4",
    "requests>=2.31.0",
    "python-multipart>=0.0.6"
]

[build-system]
requires = ["hatchling"]
build-backend = "hatchling.build"

[tool.uv]
dev-dependencies = [
    "pytest>=7.0.0",
    "pytest-asyncio>=0.21.0",
    "httpx>=0.24.0"
]
\end{verbatim}

