% Chapter 4

\chapter{User Manual}\doublespacing % Main chapter title

\label{Chapter4} % For referencing this chapter elsewhere, use \ref{Chapter4}

% Header formatting is now handled in main.tex

%----------------------------------------------------------------------------------------
%	SECTION 1
%----------------------------------------------------------------------------------------

\section{User Manual}

This chapter provides comprehensive guidance for using the SSL Automation Tool, including installation procedures, operation instructions, and troubleshooting guidelines.

\subsection{System Requirements}

Before using the SSL Automation Tool, ensure your system meets the following requirements:

\subsubsection{For End Users}
\begin{itemize}
    \item Modern web browser (Chrome 90+, Firefox 88+, Safari 14+, Edge 90+)
    \item Internet connection for accessing Let's Encrypt and DNS provider APIs
    \item GoDaddy account with API credentials (for DNS validation)
    \item Domain administrative access for certificate deployment
\end{itemize}

\subsubsection{For System Administrators}
\begin{itemize}
    \item Docker 20.10+ and Docker Compose 2.0+
    \item Linux-based server (Ubuntu 20.04+ recommended)
    \item Minimum 2GB RAM and 10GB storage space
    \item Network access to external APIs (Let's Encrypt, DNS providers)
\end{itemize}

\subsection{Installation Guide}

\subsubsection{Quick Start with Docker}

1. Clone the project repository:
\begin{verbatim}
git clone <repository-url>
cd ssl-automation-tool
\end{verbatim}

2. Start the services using Docker Compose:
\begin{verbatim}
docker compose up -d
\end{verbatim}

3. Access the application at \texttt{http://localhost:3000}

\subsubsection{Development Setup}

For developers who want to run the system in development mode:

1. \textbf{Backend Setup}:
\begin{verbatim}
cd ssl-automation-backend
pip install uv
uv sync
uv run uvicorn app.main:app --host 0.0.0.0 --port 8000 --reload
\end{verbatim}

2. \textbf{Frontend Setup}:
\begin{verbatim}
cd ssl-automation-frontend
pnpm install
pnpm dev
\end{verbatim}

\section{Operations Manual / Menu Explanation}

\subsection{Main Dashboard}

Upon accessing the application, users are presented with the SSL Certificate Dashboard containing the following sections:

\subsubsection{Statistics Cards}
\begin{itemize}
    \item \textbf{Total Certificates}: Shows the total number of managed certificates
    \item \textbf{Valid Certificates}: Displays count of currently valid certificates
    \item \textbf{Expiring Soon}: Alerts for certificates expiring within 30 days
    \item \textbf{Invalid Certificates}: Count of expired or invalid certificates
\end{itemize}

\subsubsection{Certificate List}
The certificate list displays all managed certificates with the following information:
\begin{itemize}
    \item \textbf{Domain}: The domain name for which the certificate was issued
    \item \textbf{Status}: Visual indicator showing certificate validity status
    \item \textbf{Issued Date}: When the certificate was generated
    \item \textbf{Expires On}: Certificate expiration date
    \item \textbf{Actions}: Download options for certificate files
\end{itemize}

\subsubsection{Action Buttons}
\begin{itemize}
    \item \textbf{Generate Certificate}: Opens the certificate generation dialog
    \item \textbf{Refresh List}: Updates the certificate list with latest information
    \item \textbf{Download Individual Files}: Downloads specific certificate components
    \item \textbf{Download ZIP Bundle}: Downloads complete certificate package
\end{itemize}

\subsection{Certificate Generation Process}

\subsubsection{Step 1: Access Generation Dialog}
Click the "Generate Certificate" button to open the certificate generation form.

\subsubsection{Step 2: Enter Domain Information}
\begin{itemize}
    \item \textbf{Domain}: Enter the target domain (e.g., example.com or *.example.com for wildcard)
    \item \textbf{Email}: Provide email address for Let's Encrypt registration
\end{itemize}

\subsubsection{Step 3: Configure DNS Provider}
\begin{itemize}
    \item \textbf{DNS Provider}: Select "GoDaddy" from the dropdown
    \item \textbf{API Key}: Enter your GoDaddy API key
    \item \textbf{API Secret}: Enter your GoDaddy API secret
    \item \textbf{Base Domain}: Specify the base domain for DNS operations
\end{itemize}

\subsubsection{Step 4: Advanced Settings (Optional)}
\begin{itemize}
    \item \textbf{TTL}: DNS record time-to-live (default: 600 seconds)
    \item \textbf{Propagation Delay}: Wait time for DNS propagation (default: 60 seconds)
\end{itemize}

\subsubsection{Step 5: Generate Certificate}
Click "Generate Certificate" to start the process. Monitor the progress indicators for:
\begin{itemize}
    \item Request validation
    \item DNS challenge setup
    \item Domain verification
    \item Certificate generation
    \item File storage completion
\end{itemize}

\subsection{Certificate Download and Deployment}

\subsubsection{Individual File Downloads}
For each certificate, users can download:
\begin{itemize}
    \item \textbf{fullchain.pem}: Complete certificate chain (certificate + intermediate CA)
    \item \textbf{privkey.pem}: Private key file (keep secure)
\end{itemize}

\subsubsection{ZIP Bundle Download}
Download a complete ZIP archive containing both certificate files for convenient deployment.

\subsubsection{Server Configuration}
Access the Documentation section for detailed server configuration guides:
\begin{itemize}
    \item Nginx SSL configuration
    \item Apache SSL setup
    \item Docker deployment strategies
    \item Troubleshooting common issues
\end{itemize}

\section{Program Specifications / Flow Charts}

\subsection{Certificate Generation Workflow}

The following flowchart illustrates the complete certificate generation process:

\begin{figure}[h]
\centering
\begin{tikzpicture}[node distance=2cm]

\node (start) [startstop] {User Initiates Certificate Generation};
\node (input) [process, below of=start] {User Provides Domain and DNS Credentials};
\node (validate) [process, below of=input] {System Validates Input Parameters};
\node (certbot) [process, below of=validate] {Certbot Initiates ACME Challenge};
\node (dns) [process, below of=certbot] {DNS Hook Creates TXT Record};
\node (verify) [process, below of=dns] {Let's Encrypt Verifies Domain};
\node (generate) [process, below of=verify] {Certificate Files Generated};
\node (store) [process, below of=generate] {Files Stored in Persistent Volume};
\node (complete) [startstop, below of=store] {Certificate Available for Download};

\draw [arrow] (start) -- (input);
\draw [arrow] (input) -- (validate);
\draw [arrow] (validate) -- (certbot);
\draw [arrow] (certbot) -- (dns);
\draw [arrow] (dns) -- (verify);
\draw [arrow] (verify) -- (generate);
\draw [arrow] (generate) -- (store);
\draw [arrow] (store) -- (complete);

\end{tikzpicture}
\caption{Certificate Generation Process Flow}
\label{fig:cert-generation-flow}
\end{figure}

\subsection{Error Handling Workflow}

The system implements comprehensive error handling at each stage:

\begin{enumerate}
    \item \textbf{Input Validation Errors}: Form validation provides immediate feedback
    \item \textbf{DNS API Errors}: Credential validation and API response handling
    \item \textbf{ACME Challenge Errors}: Retry mechanisms with exponential backoff
    \item \textbf{Certificate Generation Errors}: Detailed error reporting and recovery guidance
\end{enumerate}

\subsection{System Architecture Flow}

\begin{figure}[h]
\centering
\begin{tikzpicture}[node distance=3cm]

\node (user) [startstop] {User Browser};
\node (frontend) [process, right of=user] {React Frontend};
\node (api) [process, right of=frontend] {FastAPI Backend};
\node (certbot) [process, below of=api] {Certbot Service};
\node (dns) [process, left of=certbot] {DNS Provider API};
\node (le) [process, right of=certbot] {Let's Encrypt};

\draw [arrow] (user) -- (frontend);
\draw [arrow] (frontend) -- (api);
\draw [arrow] (api) -- (certbot);
\draw [arrow] (certbot) -- (dns);
\draw [arrow] (certbot) -- (le);

\end{tikzpicture}
\caption{System Architecture Data Flow}
\label{fig:system-architecture-flow}
\end{figure}

\section{Drawbacks and Limitations}

\subsection{Current System Limitations}

\begin{enumerate}
    \item \textbf{DNS Provider Support}
    \begin{itemize}
        \item Currently limited to GoDaddy DNS provider only
        \item No support for other major providers (Cloudflare, AWS Route53, etc.)
        \item Manual configuration required for each DNS provider
    \end{itemize}
    
    \item \textbf{Authentication and Security}
    \begin{itemize}
        \item No built-in user authentication system
        \item API credentials handled client-side only
        \item No role-based access control implementation
    \end{itemize}
    
    \item \textbf{Certificate Management}
    \begin{itemize}
        \item No automatic certificate renewal scheduling
        \item No certificate revocation capabilities
        \item Limited certificate metadata tracking
    \end{itemize}
    
    \item \textbf{Scalability Constraints}
    \begin{itemize}
        \item Single-instance deployment architecture
        \item No database integration for certificate metadata
        \item Limited concurrent request handling
    \end{itemize}
\end{enumerate}

\subsection{Technical Limitations}

\begin{enumerate}
    \item \textbf{Dependency on External Services}
    \begin{itemize}
        \item Reliance on Let's Encrypt service availability
        \item DNS provider API rate limiting
        \item Internet connectivity requirements
    \end{itemize}
    
    \item \textbf{Error Recovery}
    \begin{itemize}
        \item Limited automated error recovery mechanisms
        \item Manual intervention required for complex failures
        \item No automatic retry for transient failures
    \end{itemize}
\end{enumerate}

\section{Proposed Enhancements}

\subsection{Short-term Improvements}

\begin{enumerate}
    \item \textbf{Multi-DNS Provider Support}
    \begin{itemize}
        \item Implement Cloudflare DNS integration
        \item Add AWS Route53 support
        \item Create abstract DNS provider interface
    \end{itemize}
    
    \item \textbf{Enhanced Security}
    \begin{itemize}
        \item Implement JWT-based authentication
        \item Add API key management system
        \item Implement request rate limiting
    \end{itemize}
    
    \item \textbf{Improved User Experience}
    \begin{itemize}
        \item Add certificate renewal notifications
        \item Implement bulk certificate operations
        \item Enhanced error messaging and recovery guidance
    \end{itemize}
\end{enumerate}

\subsection{Long-term Enhancements}

\begin{enumerate}
    \item \textbf{Enterprise Features}
    \begin{itemize}
        \item Multi-tenant architecture
        \item Advanced monitoring and analytics
        \item Integration with CI/CD pipelines
    \end{itemize}
    
    \item \textbf{Automation Improvements}
    \begin{itemize}
        \item Automatic certificate renewal scheduling
        \item Certificate lifecycle automation
        \item Integration with certificate deployment tools
    \end{itemize}
    
    \item \textbf{Scalability Enhancements}
    \begin{itemize}
        \item Database integration for metadata management
        \item Microservices architecture implementation
        \item Horizontal scaling capabilities
    \end{itemize}
\end{enumerate}

\section{Conclusions}

The SSL Automation Tool successfully addresses the primary challenge of automating SSL certificate generation and management through Let's Encrypt integration. The system provides:

\begin{itemize}
    \item Streamlined certificate generation with DNS-01 validation
    \item User-friendly web interface with comprehensive documentation
    \item Robust backend API with proper error handling
    \item Container-based deployment for easy scaling
    \item Extensible architecture for future enhancements
\end{itemize}

The project demonstrates effective integration of modern web technologies (React 19, Next.js 15, FastAPI) with established certificate management tools (Certbot, Let's Encrypt) to create a practical solution for SSL certificate automation.

Future development should focus on expanding DNS provider support, implementing authentication mechanisms, and adding advanced monitoring capabilities to create a comprehensive enterprise-grade certificate management platform.
